\documentclass{report}
\begin{document}

\title{Documentation tool for dotty using Tasty files}
\author{Bryan Abate}
%\\Supervised by }

\maketitle

\begin{abstract}
The current documentation tool (dottydoc) relies on compiler internals and low level code. The tool introduced here aims to build a program not dependant on compiler internals but instead use Tasty files which are output when a Scala program is compiled.
\end{abstract}

\tableofcontents

\chapter{Introduction}
In dotty the documentation generation tool is called \texttt{dotty-doc}. However the tool is flawed in many ways, these flaws include:
\begin{itemize}
    \item Reliance on compiler internals
    \item Low level and unecessarily complex code
    \item Not maintened and not documented
    \item Not flexible in its output, it outputs html/css in a hard to modify way
    \item Malformed type output such as \texttt{[32m"getOffset"[0m}
    \item Lacks features, especially:
    \begin{itemize}
        
        \item 
    \end{itemize}
\end{itemize}
The tool introduced here aims to address all these issues.

\chapter{Features}

\chapter{Output format}

\chapter{Architecture}

\chapter{Problems encountered}

\chapter{Further work}

\chapter{Conclusion}

\end{document}